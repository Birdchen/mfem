\documentclass{article}
\usepackage{amsmath}
\usepackage{amssymb}
\usepackage{amsfonts}
\usepackage{pgf}

%=============================================================================
%                              Useful Commands
%=============================================================================
 
\newcommand{\refEq}[1]{(\ref{eq:#1})}
\newcommand{\refChap}[1]{Chapter~\ref{chap:#1}}
\newcommand{\refFig}[1]{Figure~\ref{fig:#1}}
\newcommand{\refSec}[1]{Section~\ref{sec:#1}}
\newcommand{\refTab}[1]{Table~\ref{tab:#1}}
\newcommand{\refApp}[1]{Appendix~\ref{app:#1}}
\newcommand{\newEq}[2]{\begin{equation} \label{eq:#1} #2 \end{equation}}
\newcommand{\cross}{\!\times\!}
\newcommand{\inner}{\!\cdot\!}
\newcommand{\Div}{\nabla\!\cdot\!}
\newcommand{\Divx}{\nabla_{\vec{x}}\!\cdot\!}
\newcommand{\Curl}{\nabla\!\times\!}
\newcommand{\Curlx}{\nabla_{\vec{x}}\!\times\!}
\newcommand{\Grad}{\nabla\!}
\newcommand{\Gradu}{\nabla_{\vec{u}}\!}
\newcommand{\Gradx}{\nabla_{\vec{x}}\!}
\newcommand{\st}{\Bigl\lvert\;}
\newcommand{\St}{\biggl\lvert\;}
%\newcommand{\Div}{\nabla\cdot}
%\newcommand{\Divx}{\nabla_{\vec{x}}\cdot}
%\newcommand{\Curl}{\nabla\times}
%\newcommand{\Curlx}{\nabla_{\vec{x}}\times}
%\newcommand{\Grad}{\nabla}
%\newcommand{\Gradu}{\nabla_{\vec{u}}}
%\newcommand{\Gradx}{\nabla_{\vec{x}}}

\providecommand{\abs}[1]{{\left\lvert#1\right\rvert}}
\providecommand{\norm}[1]{{\left\lVert#1\right\rVert}}

\def\Hdiv{$H(Div)$ }
\def\Hcurl{$H(Curl)$ }

%\pgfdeclareimage[width=5in]{pen_fun}{images/pen_fun_and_deriv}

%=============================================================================

\title{MFEM Electromagnetics Mini Applications}
\author{The MFEM Team}

\begin{document}

\maketitle

\section{Electromagnetics}

The equations describing electromagnetic phenomena are know
collectively as the ``Maxwell Equations''.  They are usually given as:
\begin{eqnarray}
\Curl\vec{H} - \frac{\partial\vec{D}}{\partial t} &=& \vec{J} \label{eq:ampere} \\
\Curl\vec{E} + \frac{\partial\vec{B}}{\partial t} &=& 0 \label{eq:faraday} \\
\Div\vec{D} &=& \rho \label{eq:gauss} \\
\Div\vec{B} &=& 0 \label{eq:divb}
\end{eqnarray}
Where equation~\refEq{ampere} can be referred to as {\em Amp\`ere's Law},
equation~\refEq{faraday} is called {\em Faraday's Law},
equation~\refEq{gauss} is {\em Gauss's Law}, and equation~\refEq{divb}
doesn't generally have a name but is related to the nonexistence of
magnetic monopoles.  The various fields in these equations are:
\begin{center}
\begin{tabular}{|l|l|l|}
\hline
Symbol & Name & SI Units \\
\hline
$\vec{H}$ & Magnetic Field & Ampere/meter \\
$\vec{B}$ & Magnetic Flux Density & Tesla \\
$\vec{E}$ & Electric Field & Volts/meter \\
$\vec{D}$ & Electric Displacement & Coulomb/meter$^2$ \\
$\vec{J}$ & Current Density & Ampere/meter$^2$ \\
$\rho$ & Charge Density & Coulomb/meter$^3$ \\
\hline
\end{tabular}
\end{center}
In the literature these names do vary, particularly those for
$\vec{H}$ and $\vec{B}$, but in this document we will try to adhere to
the convention laid out above.

Generally we also need constitutive relations between $\vec{E}$ and
$\vec{D}$ and/or between $\vec{H}$ and $\vec{B}$.  These relations
start with the definitions:
\begin{eqnarray}
\vec{D} &=& \epsilon_0\vec{E} + \vec{P}\\
\vec{B} &=& \mu_0\vec{H} + \vec{M}
\end{eqnarray}
Where $\vec{P}$ is the {\em polarization density}, and $\vec{M}$ is
the {\em magnetization}.  Also, $\epsilon_0$ is the {\em permittivity
  of free space} and $\mu_0$ is the {\em permeability of free space}
which are both constants of nature.  In many common materials the
polarization density can be approximated as a scalar multiple of the
electric field i.e. $\vec{P}=\epsilon_0\chi\vec{E}$, where $\chi$ is
called the {\em electric suscepctibility}.  In such cases we usually
use the relation $\vec{D}=\epsilon\vec{E}$ with
$\epsilon=\epsilon_0(1+\chi)$ and call $\epsilon$ the {\em
  permittivity} of the material.

The nature of magnetization is more
complicated but we will take a very simplified view which is valid in
many situations.  Specifically, we will assume that either $\vec{M}$
is proportional to $\vec{H}$ yielding the relation
$\vec{B}=\mu\vec{H}$ where $\mu=\mu_0(1+\chi_M)$ and $\chi_M$ is the
{\em magnetic susceptibility} or that $\vec{M}$ is a constant.  The
former case pertains to both diamagnetic and paramagnetic materials
and the later to ferromagnetic materials.

Finally we should note that equations~\refEq{ampere} and \refEq{gauss}
can be combined to yield the equation of charge continuity
\[\frac{\partial\rho}{\partial t}+\Div\vec{J} = 0\]
which can be important in plasma physics and magnetohydrodynamics (MHD).

\subsection{Static Fields}
\subsubsection{Electrostatics}

Electrostatic problems come in a variety of subtypes but they all
derive from Gauss's Law and Faraday's Law (equations~\refEq{gauss} and
\refEq{faraday}).  When we assume no time variation, Faraday's Law
becomes simply $\Curl\vec{E}=0$. This suggests that the electric field
can be expressed as the gradient of a scalar field which is
traditionally taken to be $-\varphi$, i.e.
\begin{equation}
\vec{E} = -\Grad\varphi \label{eq:gradphi}
\end{equation}
where $\varphi$ is called the {\em electric potential} and has units
of Volts in the SI system.  Inserting this definition into
equation~\refEq{gauss} gives:
\begin{equation}
-\Div\epsilon\Grad\varphi = \rho \label{eq:poisson}
\end{equation}
which is {\em Poisson's equation} for the electric potential.  Where,
clearly, we have assumed a linear constitutive relation between
$\vec{D}$ and $\vec{E}$.  If this relation happens to be nonlinear
then Poisson's equation would need to be replaced with a more
complicated nonlinear expression.

The solutions to equation~\refEq{poisson} are non unique because they
can be shifted by any additive constant.  This means that we must
apply a Dirichlet boundary condition at at least one point in the
problem domain in order to obtain a solution.  Typically this point
will be on the boundry but it need not be so.  Such a Dirichlet value
is equivalent to fixing the voltage (aka potential) at one or more
locations.  Additionally, this equation admits a normal derivative
boundary condition.  This means setting $\hat{n}\cdot\vec{D}$ to a
prescribed value on some portion of the boundary.  This is equivalent
to defining a surface charge density on that portion of the boundary.

\subsubsection{Magnetostatics}

Magnetostatic problems arise when we assume no time variation in
Amp\`ere's Law (equation~\refEq{ampere}) which leads to:
\[\Curl\vec{H}=\vec{J}\]
In this case we'll assume a somewhat more general constitutive
relation between $\vec{H}$ and $\vec{B}$:
\[\vec{B}=\mu\vec{H}+\vec{M}\]
Notice that we use $\mu$ rather than $\mu_0$.  This allows for
paramagnetic and/or diamagnetic materials defined through $\mu$ as
well as ferromagnetic materials represented by $\vec{M}$.  This choice
yields:
\[\Curl\mu^{-1}\vec{B}=\vec{J}+\Curl\mu^{-1}\vec{M}\]
Of course a nonlinear constitutive relation is possible and would lead
to a rather complicated nonlinear equation for $\vec{B}$.

To reach an equation we can solve in the linear case we need to make
use of equation~\refEq{divb} which implies that $\vec{B}=\Curl\vec{A}$
for some potential $\vec{A}$ which is called the {\em magnetic vector
  potential}.  This gives the final form of the equation:
\begin{equation}
\Curl\mu^{-1}\Curl\vec{A}=\vec{J}+\Curl\mu^{-1}\vec{M}
\end{equation}
Once again the potential is non unique so we must apply Dirichlet
boundary conditions in order to arrive at a solution for $\vec{A}$.   

more on appropriate boundary conditions\ldots

more on magnetic scalar potential\ldots

\subsubsection{Statics Miniapp}

Possible sources:
\begin{itemize}
\item Charge Density $\rho$
\item Current Density $\vec{J}$
\item Magnetization $\vec{M}$
\item Surface Charge Density $\sigma$
\item Surface Current Density $\vec{K}$
\end{itemize}

Additional Boundary Conditions:
\begin{itemize}
\item Voltage at at least one point
\item Magnetic Vector Potential on some region (restrictions???)
\item Magnetic Scalar Potential at at least one point (we may be able
  to choose this for the user)
\end{itemize}

\end{document}
